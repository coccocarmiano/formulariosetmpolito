\documentclass[12pt]{extarticle}
    \title{FormularioSetm}
    \author{{Cocco}}
    \date{}
\usepackage[margin=2cm]{geometry} 
\begin{document}
\thispagestyle{empty}
{\bf \huge Misure}
\section{Errori}.

%{\bf Media:} $ \displaystyle \mu = \frac{1}{N}\sum_i x_i $ \ \ \ {\bf Varianza:} $\displaystyle \sigma^2 = \frac{1}{N-1}\sum (x_i - \mu)^2 $ 

{\bf Incertezza Tipica:} $\pm (\% lettura + \% fondo-scala) $

{\bf Strumento di classe x:} lo strumento ha una precisione pari ad x\% del fondo-scala

%L'incertezza va riportata con {\bf al più} due cifre significative. (e.g.: $ R = (4700 \pm 47) \Omega $

{\bf Errore Qualunque (Derivate Parziali):} $\displaystyle y = f(x_1, x_2, ...) \to \delta y  = \sum_i \Bigg |\frac{\partial f(\vec{x})}{\partial x_i}\Bigg |_{x=x_{mis}} \delta x_i $

{\bf Errore Relativo:} $y$ misura $\displaystyle\to \epsilon_r = \frac{\delta y}{y} \ \ (\epsilon_{r, \%} = 100\cdot\epsilon_r)$

{\bf Somma/Differenza:} $y = a \pm b \to \delta y = \delta a + \delta b$


{\bf Prodotto/Divisione:} $\displaystyle y = a\cdot b^{\pm 1} \to \epsilon_y = \epsilon_a + \epsilon_b$

{\bf Potenza/Radice:} $\displaystyle y = x^{\pm n} \to \epsilon_y = n^{\pm 1}\epsilon_x$

\section{Attrezzature}. 

%{\bf Full-Scale Range:} $ FSR = V_{max} - V_{min} $ 

% {\bf Quantizzazione su N bit:} $ N = log_2(M) $

% {\bf Uscita Ricostruita:} $\displaystyle y = \sum_{i=0}^{n-1} 2^i\cdot B_i \ \ (B = 0 \lor B = 1)$ \ \ \ {\bf MSB:} $ B_{n-1} $ \ {\bf LSB:} $ B_0 $ 

%{\bf Teorema del Campionamento (B banda del segnale):} $ f_c \ge 2\cdot B $

{\bf Errore di Lettura (Parallasse):} valore di una "tacchetta"

{\bf Prodotto Banda Guadagno:} $ B\cdot t_{salita} = 0.35 $ (varia col modello dell'oscilloscopio)

{\bf Relazione Salita-Visualizzato:} $\displaystyle t_{s, visualizzato}^2 = t_{s, segnale}^2 + t_{s, circuito}^2 + t_{s, oscilloscopio}^2$

Solitamente il $t_{so}$ è trascurabile.

{\bf Valor-Medio:} $\displaystyle v_m = \frac{1}{T} \int_{t_0}^{t_0+T} v(t)dt $ {\bf Valore Efficace:} $\displaystyle v_{rms}= \sqrt{\frac{1}{T} \int_{t_0}^{t_0+T} v^2(t)dt}$

{\bf Duty-Cycle:} $\displaystyle D = \frac{t_{alto}}{T}$ 

{\bf Sensibilità:} $\displaystyle (k_v)$ "altezza", solitamente in mV, di un "quadratino".

{\bf Lettura:} $\displaystyle V_{pp} $ del segnale, ottenuta come $\displaystyle k_v\cdot n_{div}, \ n_{div}$ altezza del segnale in "quadratini"

{\bf Voltmetro a Doppia Rampa:} $\displaystyle V_x = -\frac{T_2}{T_1}V_{rif}$ \ {\bf Incertezza di Quantizzazione:} $\delta f_q = \frac{1}{T_{mis}}$


{\bf Frequenzimetro a Misura Diretta:} $\displaystyle f_x = \frac{1}{t_x} = \frac{n}{T_c}, \ \ T_x = nT_c$

{\bf Risoluzione:} $\displaystyle \delta f_x = \frac{1}{T_c}$ \ \ \ {\bf Risoluzione Relativa:} $\displaystyle \frac{\delta f_x}{f_x} = \frac{1}{n}$

{\bf Scelta di $T_1$:} $n(t)$ ruomore di periodo $T \to T_1 = T$ (spesso 50 Hz)

{\bf Riscaldamento di un Resistore:} $\displaystyle T_{fin} - T_{amb} = R_{termica}\cdot P_{dissipata}$

{\bf Ponte di Wheatstone:} All'equilibrio, la ddp ai due nodi centrali è nulla 

{\bf Potenza} $\displaystyle P = \frac{v_{eff}^2}{R} $, per segnali sinusoidali $\displaystyle v_{eff} = \frac{A}{\sqrt{2}}$

%{\bf Valor-Medio Segnale Raddrizzato (Singola/Doppia Semionda):}  $\displaystyle v_m = \frac{2\cdot V_p}{\pi} \neq v_{rms}$ 

%{\bf Valore Efficace Segnale Raddrizzato:} $\displaystyle v_{eff} = \frac{V_p}{\sqrt{2}} = k_s\cdot\frac{2\cdot V_p}{\pi}$

{\bf Costante Strumentale:} $\displaystyle k_s = 1.11 $ (semionda doppia) | $\displaystyle k_s = 2.22 $ (semionda singola)

Gli strumenti spesso riportano il valore efficace di un segnale. Questo valore è pari alla

costante strumentale moltiplicata per valor medio del segnale. 

{\bf !} Nel calcolo del valor medio con voltmetro a semionda semplice della semionda semplice la 

parte negativa del segnale viene azzerata.
\newpage

{\bf Condensatore in ingresso / Voltmetri TRMS:} I voltmetri TRMS restituiscono la lettura

reale del valore efficace di un segnale. Tuttavia vengono spesso {\bf accoppiati in AC} 

(filtro sulla componente DC), riportando così il {\bf valore efficace del segnale originale

traslato in basso del suo valor medio}.

{\bf Valori Efficaci noti: solo} per ampiezze {\bf simmetriche} ($-V_p \to V_p, \ -5V \to 5V, ...$):

{\bf Onda Quadra:} $\displaystyle v_{eff} = V_p$ \ \ \ \ {\bf Sinusoidi:} $\displaystyle v_{eff} = \frac{V_p}{\sqrt{2}}$ \ \ \ \ {\bf Triangolari:} $\displaystyle v_{eff} = \frac{V_p}{\sqrt{3}}$


{\bf Circuito equivalente d'ingresso DSO:} è rappresentato dal parallelo tra un condensatore

(decine di $\displaystyle pF $) e una resistenza (solitamente $\displaystyle 1M\Omega $).
\\\\\\

{\bf \Huge Elettronica}
\setcounter{section}{0}
\section{Segnali}.



{\bf Signal-to-Noise Ratio:} $\displaystyle SNR = \frac{P_{segnale}}{P_{rumore}} = \frac{v^2}{n^2}$ 

{\bf SNR in dB:} $\displaystyle SNR_{dB} = 10log_{10}(SNR) = 20log_{10}(\frac{v}{n})$

{\bf Errore di Quantizzazione:} $\displaystyle \epsilon_q = \frac{S}{2^{N+1}} = \frac{1}{2}LSB $, con S dinamica del segnale

\section{Diodi}.

{\bf Polarizzazione Diretta:} $\displaystyle v_D > 0, i_D \to \infty $, curva verticale per $v_D > V_\gamma \simeq 0.6-0.7 \ V$

{\bf Polarizzazione Inversa:} $\displaystyle v_D < 0, i_D$ satura a valori piccoli e negativi (pA-fA)

{\bf Diodo Reale:} $\displaystyle i_D = I_s(e^{\frac{v_D}{\eta V_T}}- 1)$

{\bf Diodo Ideale:} $i_D > 0 \to v_D = 0, ON \ \big | \ v_D < 0 \to i_D = 0, OFF $ (nel circuito: generatore)

{\bf Diodo Semi-Ideale:} $i_D > 0 \to v_D = V_\gamma, ON \ \big | \ v_D < V_\gamma \to i_D = 0, OFF$ (nel circuito: corto)

{\bf Resistenza di un Diodo (Piccolo Segnale)}: $\displaystyle g_D = \frac{1}{r_D} = \frac{I_D}{\eta V_T}$

\section{Transistors}.

{\bf Corrente di Gate:} $i_G = 0$ in condizioni statiche per NMOS e PMOS. (BJT $> 0$)

{\bf nMOS:} $\displaystyle v_{GS}, \ v_{DS}$ \ \ \ {\bf pMOS:} $\displaystyle v_{SG}, \ v_{SD}$

{\bf "Trucco" Mnemonico:} Il {\bf S}ource è sempre dov'è la corrente. Il {\bf G}ate sempre la sbarra.

Il {\bf D}rain il rimanente. Per capire se usare $v_{GS}$ o $v_{SG}$, bisogna posizionare due tensioni verso

l'alto, una tra le due "gambe" del transistor (che sarà $v_{DS}$ o $v_{SD}$) e una tra Gate e Source

(che sarà $v_{GS}$ o $v_{SG}$), ricordando che $v_{XY}$ è una tensione con la punta in X e la coda in Y.

{\bf Condizioni Saturazione:} $\displaystyle v_{GS/SG} > V_{TH}, \ v_{DS/SD} > v_{GS/SG} - V_{TH} $

{\bf Corrente di Drain (xMOS):} 

$\displaystyle OFF \to i_D = 0 \ \big | \  ON \to i_D = \beta v_{DS}(v_{GS}-V_{TH}-\frac{v_{DS}}{2}) \ \big | \ SAT. \to i_D = \frac{\beta}{2}(v_{GS}-V_{TH})^2(1+\lambda v_{DS})$

{\bf Resistenze:} $\displaystyle g_m = \sqrt{2I_D\beta} = \beta(v_{GS}-v_{TH})\ \ \ g_o = \lambda I_D$

\section{Stadi Amplificatori}.

{\bf Tensione Ideale:} $R_{in} \to \infty, \ R_{out} \to 0 $ \ \ {\bf Corrente Ideale:} $R_{in} \to 0, \ R_{out} \to \infty $ 

{\bf Transconduttanza:} cioè $ i_{out} = g_mv_{gs} $, {\bf ideale} se $R_{in} \to \infty, \ R_{out} \to \infty$

{\bf Transresistenza:} cioè $ v_{out} = R_mi_s$, {\bf ideale} se $R_{in} \to 0, R_{out} \to 0$ 

{\bf Efficienza Amplificatore Potenza:} $\displaystyle \eta = \frac{P_{out}}{P_{al}+P_{in}} \simeq \frac{P_{out}}{P_{al}}$ 

{\bf Common Source:} $ A_v < 0, \ R_{in} \to \infty \ R_{out} = $ finita 

{\bf Common Drain:} $ A_v < 1, \ R_{in} \to \infty \ R_{out} \simeq \frac{1}{g_m}$	

{\bf Common Gate:} $ A_v \simeq g_m(R || r_o) \ R_{in} \simeq \frac{1}{g_m} \ R_{out} = R || r_o$

{\bf Effetti di Carico:} Uno stadio si comporta come un amplificatore con generatore pilotato.

Trovata $R_{in}$ ed $R_{out}$, rappresenta il blocco centrale di un circuito di mezzo tra un 

generatore in ingresso $v_{in}$ con resistenza $R_S$ e un'uscita $v_{out}$ con resistenza $R_L$. Si puo

ricavare l'uscita $v_{out}$ in funzione dell'ingresso $v_{in}$ (e quindi la funzione di trasferimento $A_{v,s}$)

eseguendo semplici partitori 

Esempio: se lo stadio è un partitore di tensione, allora $\displaystyle v_{out} = v_{in}\frac{R_{in}}{R_{in}+R_S}A_v\frac{R_L}{R_L+R_{out}}$


\section{Amplificatori Operazionali}.

{\bf Relazione Fondamentale (ideale):} $\displaystyle v_d = 0 \to v^+ = v^- \ \ \ i^+ = i^- = 0$

{\bf Amplificatore di Tensione:} $\displaystyle A_v = 1+\frac{R_2}{R_1}, R_{in} \to \infty, R_{out} \to 0$

{\bf Amplificatore di Transconduttanza:} $\displaystyle G_m = \frac{1}{R} \ $(quella in retroazione)$, R_{in} \to \infty, R_{out} \to \infty$

{\bf Amplificatore di Transresistenza: } $\displaystyle R_m = R \ $(quella in retroazione)$, R_{in} \to 0, R_{out} \to 0$

{\bf Amplificatore di Corrente:} $\displaystyle A_i = 1+\frac{R_2}{R_1}, R_{in} \to 0, R_{out} \to \infty$

{\bf Voltage-Follower:} $\displaystyle v_{out} = v_{in}, R_{in} \to \infty, R_{out} \to 0$

{\bf Invertente:} $\displaystyle A_v = -\frac{R_2}{R_1}, R_{in} = R_1, R_{out} \to 0$

{\bf Esponenziale:} Diodo su $R_1$\ \ {\bf Logaritmico:} Diodo su $R_2$ \ {\bf Integratore:} Condensatore su $R_2$

{\bf Derivatore:} Condensatore su $R_1$ (Per tutti e 4: $R_{out} = 0$)


{\bf Sommatore Generalizzato:} $\displaystyle v_{out} = \frac{\sum_{i=0}^M G_{i^-} + G_f}{\sum_{i=0}^N G_{i^+}} \sum_{i=1}^N \frac{G_{i^+}}{G_f}v_{i^+}-\sum_{i=1}^M\frac{G_{i^-}}{G_f}v_{i^-}$ 

(Non usare questa formula, meglio sovrapposizione e/o Millman)

{\bf Common-Mode Rejection Ratio (CMRR):} $\displaystyle \bigg |\frac{A_d}{A_{cm}} \bigg |, \ v_{cm} = \frac{v^++v^-}{2}, \ v_d = v^+-v^-$


\section{Limiti Amplificatori Operazionali}.

{\bf Circuito Eq. in Linearità:} $\displaystyle v_{out} = A_dv_d +A_{cm}v_{cm}+A_{ps}v_{ps}$

{\bf Amplificazione Differenziale Finita:} $\displaystyle A_v = \frac{\beta A_d}{1+\beta A_d}\frac{1}{\beta} = \frac{1}{\beta}$ se $A_d \to \infty$

{\bf Prodotto Banda-Guadagno} $\displaystyle B = \beta f_T$ \ \ {\bf Parametro Beta:} $\displaystyle \beta = -\frac{v_d}{\hat{e}} = -\frac{v_d}{A_dv_d + A_{cm}v_{cm} + ...}$

{\bf Casi Particolari:} Ad esempio operazionale ideale con parametri canonici 

($A_{cm} \simeq 0,\ R_{in} \to \infty, ...$) allora $\displaystyle \beta = 1 + \frac{R_2}{R_1}$, per invertenti e non-invertenti.

(Esempio: se invertente con $A_d = -9$, allora $\frac{R_2}{R_1} = 9$ da cui $\beta = 1 + \frac{R_2}{R_1} = 10$)

{\bf Slew-Rate:} $\displaystyle \bigg |\frac{dv}{dt}\bigg |_{max} \leq \bigg |SR\bigg |.$ Ipotizziamo sinusoide in ingresso amplificata all'uscita.

$\displaystyle \bigg |\frac{\partial}{\partial{t}} V_{in}A_vsin(\omega t)\bigg |_{max}\leq \bigg |SR\bigg |$. Quindi: $\displaystyle \bigg |\omega A_vV_{in}\bigg | = \bigg |2\pi fA_vV_{in}\bigg | \leq \bigg | SR \bigg |$

{\bf Offset di Tensione:} $V_{OFF}$ collegata al morsetto +

{\bf Correnti di Polarizzazione:} Corrente $I^+/I^- = I_{BIAS}+\frac{I_{OFF}}{2}$ ai due morsetti, uscente

$\displaystyle I_{BIAS} = \frac{I^++I^-}{2}, \ I_{OFF} = I^++I^-$

\section{Comparatori}.

Ottenuti con retroazione positiva.

{\bf Soglie (Invertente):} $\displaystyle V_{S1} = V_S\frac{R_2}{R_2+R_1}+V_{OH}\frac{R_1}{R_1+R_2} \ \ V_{S2} = V_{S1} $ ma con OL, $V_S$ sul '+'

{\bf Soglie (Non Invertente):} $\displaystyle V_{S1} = V_S(1+\frac{R1}{R2}) - V_{OL}\frac{R1}{R2}, \ V_{S2} = V_{S1}$ ma con OH, $V_S$ sul '-'

{\bf Comparatori Reali:} $\displaystyle V_{S1/2, reale} = V_{S1/2, ideale}-V_{OFF}$

\section{Oscillatori}

da fare

\end{document}

